\documentclass[a4paper,12pt]{article}
\usepackage[margin=1.00in]{geometry}

\title{Problem 1\\
\large Course - SOEN 6011, Professor - Pankaj Kamthan}
\author{Shagun Shagun, ID - 40138455}
\date{}
\begin{document}

\maketitle %To display the title in the document.

\section{\large Function 5 : Power Function}

\subsection{Description}
A power function is of the form:
\begin{equation} \label{Power_func}
	f(x) = ab^x
\end{equation}
where x is a real number, a and b are constants.


\subsection{Domain}
The domain is set of all real numbers,  (- $\infty$,$\infty$)

\subsection{Co-domain}
The co-domain is also set of all real numbers.


\subsection{Characteristics of Power Function.}
\begin{enumerate}
\item For any exponential function, the domain is the set of all real number, however range is bounded by the horizontal symptote of the graph of f(x).

\item The behaviour of power function effects the exponential growth and decay.
\item When b greater than 1, the graph accelerates towards y-axis contributing to exponential growth.
\item When b greater than 1 and less than 0, the graph decreases towards y-axis contributing to exponential decay.
\item When modeling real-world situations with an exponential function, the domain and range can be limited to numbers that make sense in the context. The domain and range can be stated using the inequalities for a continuous interval in these cases.

\end{enumerate}

\end{document}
